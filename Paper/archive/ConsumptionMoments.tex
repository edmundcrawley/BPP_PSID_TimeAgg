% !TeX spellcheck = en_GB
\documentclass[12pt,pdftex,letterpaper]{article}
%            \usepackage{setspace}
\usepackage[dvips,]{graphicx} %draft option suppresses graphics dvi display
%            \usepackage{lscape}
%            \usepackage{latexsym}
%            \usepackage{endnotes}
%            \usepackage{epsfig}
%           \singlespace
\setlength{\textwidth}{6.5in}
\setlength{\textheight}{9in}
\addtolength{\topmargin}{-\topmargin} 
\setlength{\oddsidemargin}{0in}
\setlength{\evensidemargin}{0in}
\addtolength{\headsep}{-\headsep}
\addtolength{\topskip}{-\topskip}
\addtolength{\headheight}{-\headheight}
\setcounter{secnumdepth}{2}
%            \renewcommand{\thesection}{\arabic{section}}
% \renewcommand{\footnote}{\endnote}
\newtheorem{proposition}{Proposition}
\newtheorem{definition}{Definition}
\newtheorem{lemma}{lemma}
\newtheorem{corollary}{Corollary}
\newtheorem{assumption}{Assumption}
\newcommand{\Prob}{\operatorname{Prob}}
\clubpenalty 5000
\widowpenalty 5000
\renewcommand{\baselinestretch}{1.25}
\usepackage{amsmath}
\usepackage{amsthm}
\usepackage{amsfonts}
\usepackage{amssymb}
\usepackage{bbm}
\newcommand{\N}{\mathbb{N}}
\newcommand{\R}{\mathbb{R}}
\newcommand{\E}{\mathbb{E}}

\begin{document}

\section{More General Consumption Processes}

Edmund Crawley\\
\today

\section{Setup}
Assume now that:
\begin{align*}
c_t dt = \int_{-\infty}^{t} f(t-s)dQ_s dt
\end{align*}
(this is a generalization of $f(t-s)=\phi$, the random walk case from BPP).

We have
\begin{align*}
\Delta \bar{c}_T &= \int_{T-1}^{T} \int_{-\infty}^{t} f(t-s) dQ_s dt - \int_{T-2}^{T-1} \int_{-\infty}^{t} f(t-s) dQ_s dt \\
&= \int_{T-1}^{T} \Big( \int_{-\infty}^{T-2} f(t-s)dQ_s + \int_{T-2}^{T-1} f(t-s)dQ_s + \int_{T-1}^{t} f(t-s)dQ_s \Big) dt \\
& \qquad -  \int_{T-2}^{T-1} \Big( \int_{-\infty}^{T-2} f(t-s)dQ_s  + \int_{T-1}^{t} f(t-s)dQ_s \Big) dt \\
&=  \int_{-\infty}^{T-2} \Big( \int_{T-1}^{T} f(t-s)dt -  \int_{T-2}^{T-1} f(t-s)dt  \Big) dQ_s \\
& \qquad +  \int_{T-2}^{T-1} \Big(\int_{T-1}^{T}f(t-s)dt - \int_{s}^{T-1}f(t-s)dt  \Big) dQ_s \\
& \qquad + \int_{T-1}^{T} \Big( \int_{s}^{T} f(t-s) dt \Big) dQ_s
\end{align*}

\section{Examples}
\subsection{Example 1: Exponential Decay}
The first example will assume exponential decay of consumption, with households eventually using up all their money
\begin{align*}
f(t-s) = \phi e^{-\phi(t-s)}
\end{align*}
This implies an annual MPC of
\begin{align*}
\text{MPC} &= \int_{0}^{1} \phi e^{-\phi t}dt \\
&= \Big[ -e^{-\phi t} \Big]^1_0 \\
&= 1-e^{-\phi} 
\end{align*}
Similarly quarterly MPC is $1-e^{-0.25\phi}$. We have:
\begin{align*}
\Delta \bar{c}_T &=   \int_{-\infty}^{T-2} \Big(e^{-\phi (T-s)}(e^{\phi}-1) -  e^{-\phi (T-1-s)}(e^{\phi}-1)  \Big) dQ_s \\
& \qquad +  \int_{T-2}^{T-1} \Big(e^{-\phi (T-s)}(e^{\phi}-1) - 1+ e^{-\phi (T-1-s)} \Big) dQ_s \\
& \qquad + \int_{T-1}^{T} \Big( 1-e^{-\phi (T-s)} \Big) dQ_s
\end{align*}

\begin{align*}
\Delta \bar{c}_T &=   \int_{-\infty}^{T-2} \Big(-e^{-\phi (T-s)}(e^{\phi}-1)^2  \Big) dQ_s \\
& \qquad +  \int_{T-2}^{T-1} \Big(e^{-\phi (T-s)}(2e^{\phi}-1) - 1 \Big) dQ_s \\
& \qquad + \int_{T-1}^{T} \Big( 1-e^{-\phi (T-s)} \Big) dQ_s
\end{align*}

\subsection{Example 2: Splurge, then constant}
This example has two parameters and will be identified by both $cov(\Delta \bar{c}_t,\Delta \bar{y}_{t+1})$ and $cov(\Delta \bar{c}_t,\Delta \bar{y}_{t-1})$. It is equivalent to an initial `splurge' followed by a fixed increase in consumption. It can be mapped to a model with durable expenditure where there is an initial large sum spent on durable goods. Formally:
\begin{align*}
f(t-s) = \psi_1 \delta_0(t-s) + \psi_2 \mathbbm{1}_{t-s>0}
\end{align*}
where $\delta_0$ is the dirac delta function.
\begin{align*}
\Delta \bar{c}_T &=  \psi_1 \Big( \int_{T-1}^{T} dQ_s- \int_{T-2}^{T-1}dQ_s \Big) \\
& \qquad + \psi_2 \Big(\int_{T-2}^{T-1} (s-(T-2))dQ_s  + \int_{T-1}^{T} (T-s)dQ_s \Big)
\end{align*}
\begin{align}
\Delta \bar{c}_T &=   \int_{T-1}^{T} \big( \psi_1 +\psi_2 (T-s)\big) dQ_s \nonumber \\
& \qquad + \int_{T-2}^{T-1} \big(\psi_2  (s-(T-2)) -\psi_1 \big) dQ_s \label{splurge_c}
\end{align}

\section{Persistence in Transitory Shock}
\subsection{Most Closely Parallel to MA(1)}
BPP add MA(1) persistence in the transitory shock. The most closely related continuous time version (that doesn't seem to relate to reality, but might work in practice), is to assume a transitory payment of 1 followed by another payment of $\theta$ exactly one year later. That is:
\begin{align*}
y_t dt &= \Big(\int_{0}^{t} dP_s \Big) dt + dQ_t + \theta dQ_{t-1}
\end{align*}
So that:
\begin{align}
\Delta \bar{y}_T  &= \Big(\int_{T-1}^{T} (T-s) dP_s + \int_{T-2}^{T-1} (s-(T-2)) dP_s\Big)  \nonumber \\
& \qquad + \Big(\int_{T-1}^{T} dQ_s - (1-\theta)\int_{T-2}^{T-1}  dQ_s - \theta\int_{T-3}^{T-2}  dQ_s \Big) \label{MA1_y}
\end{align}
\subsection{Combining persistence with splurge}
Here I calculate the moments related to a model of equation \ref{splurge_c} extended to for permanent shocks, combined with the income process \ref{MA1_y}.
\begin{align*}
cov(\Delta \bar{c}_T, \Delta \bar{y}_{T-1}) &= \int_{T-2}^{T-1} (T-1-s)(\phi_2 (s- (T-2)) - \phi_1)dP_s dP_s \\
& \qquad + \int_{T-2}^{T-1} \big(\psi_2  (s-(T-2)) -\psi_1 \big) dQ_s dQ_s \\
&= \frac{1}{6}\phi_2 \sigma^2_P -\frac{1}{2}\phi_1 \sigma^2_P +\frac{1}{2}\psi_2 \sigma^2_Q - \psi_1 \sigma^2_Q \\
cov(\Delta \bar{c}_T, \Delta \bar{y}_{T}) &= \int_{T-1}^{T} (\phi_1+\phi_2(T-s))(T-s) dP_s dP_s + \int_{T-2}^{T-1} (\phi_2 (s- (T-2)-\phi_1)(s-(T-2)) dP_s dP_s \\
& \qquad \int_{T-1}^{T} (\psi_1+\psi_2(T-s)) dQ_s dQ_s - (1-\theta)  \int_{T-2}^{T-1} (\psi_2 (s- (T-2))-\psi_1) dQ_s dQ_s \\
&= \frac{1}{2}\phi_1 \sigma^2_P +\frac{1}{3}\phi_2 \sigma^2_P +\frac{1}{3}\phi_2 \sigma^2_P - \frac{1}{2}\phi_1 \sigma^2_P \\
& \qquad + \psi_1 \sigma^2_Q + \frac{1}{2}\psi_2 \sigma^2_Q - (1-\theta)\Big(\frac{1}{2}\psi_2 - \psi_1\Big)\sigma^2_Q\\
&= \frac{2}{3} \phi_2 \sigma^2_P + (2-\theta)\psi_1 \sigma^2_Q +\frac{1}{2}\theta \psi_2 \sigma^2_Q \\
cov(\Delta \bar{c}_T, \Delta \bar{y}_{T+1}) &=  \int_{T-1}^{T}  (\phi_1+\phi_2(T-s)) (s-(T-1)) dP_s dP_s \\
& \qquad -(1-\theta) \int_{T-1}^{T}  (\psi_1+\psi_2(T-s)) dQ_s dQ_s - \theta  \int_{T-2}^{T-1} (\psi_2 (s- (T-2))-\psi_1) dQ_s dQ_s \\
&= \frac{1}{2}\phi_1 \sigma^2_P +\frac{1}{6} \phi_2 \sigma^2_P  -(1-\theta) (\psi_1 + \frac{1}{2}\psi_2)\sigma^2_Q -\theta(\frac{1}{2}\psi_2 - \psi_1)\sigma^2_Q \\ 
&= \frac{1}{2}\phi_1 \sigma^2_P +\frac{1}{6} \phi_2 \sigma^2_P -(1-2\theta)\psi_1 \sigma^2_Q -\frac{1}{2}\psi_2 \sigma^2_Q \\
cov(\Delta \bar{c}_T, \Delta \bar{y}_{T+2}) &=  -\theta \int_{T-1}^{T} (\psi_1 + \psi_2(T-s))dQ_sdQ_s \\
&= -\theta (\psi_1 +\frac{1}{2}\psi_2)\sigma^2_Q
\end{align*}

\subsection{Transitory Shock that Lasts for Time $\tau$}
Now suppose the transitory shock is not instantaneous but instead spread out over a period of time $\tau$. That is instantaneous income is now:
\begin{align*}
y_t dt &= \Big(\int_{0}^{t} dP_s \Big) dt +\Big(\int_{t-\tau}^{t} \frac{1}{\tau} dQ_s \Big)dt
\end{align*}

\end{document}


