\subsection{Exponential Decay in the Transitory Consumption Response} \label{exp_decay_appendix}
The work in this paper, along with numerous other natural experiments, suggest that the consumption response to a transitory income shock over three months to a year is large. This behavior is incompatible with the consumption moving as a random walk, at least if the household budget constraint is to hold. Motivated by this, as well as the unexplained differences in estimate of $\phi$ between the time aggregated and BPP models, here I extend the model to allow for exponential decay in consumption response to transitory shocks.

In this model, permanent income is modeled exactly as in the main paper. However, both transitory income shocks and the consumption response decay exponentially, albeit at different rates. A shock to transitory income follows the path:
\begin{align*}
f(t) = \frac{\Omega}{1-e^{-\Omega}} e^{-\Omega t}
\end{align*}
Where $\Omega$ is the rate of decay, and the denominator is chosen such that a shock of size one increases income in the following year by one.

Similarly, a the consumption path following a transitory income shock follows the path:
\begin{align*}
g(t) = \frac{\psi \theta}{1-e^{-\theta}} e^{-\theta t}
\end{align*}
So that consumption decays at a rate $\theta$, and the increase in consumption in the year following a unit transitory income shock is $\psi$. The relevant moments for this model are calculated at the end of this appendix in section \ref{exp_decay_moments_appendix}.

Table \ref{table:ExpIncDecay} shows the estimates for the insurance parameters, as well as the rates of decay, for this model.  The transitory insurance parameter, $\psi$, is slightly lower than in the main paper, but in the same ballpark at 0.19. The permanent insurance parameter, $\phi$, is significantly below the estimate in the main paper, and statistically no different from the estimate for $\psi$. This deepens the puzzle that $\phi$ appears to be too low. The estimates for $\Omega$ and $\theta$ suggest the half life of a transitory income shock to be close to one month, while the consumption response to this income shock has a half life close to one year.

\input ../Code/Tables/ExpIncDecay.tex

\subsubsection{Moment in the Exponential Decay Model} \label{exp_decay_moments_appendix}
To keep the notation manageable, the moments calculated below are for the transitory components in income and consumption only. The full model adds in the same permanent income and consumption as the main paper (permanent and transitory shocks are independent, so the variances and covariances are additive).\\

\textit{Exponentially Decaying Income Process}\\

A shock to transitory income decays exponentially according to the function:
\begin{align*}
f(t) = \frac{\Omega}{1-e^{-\Omega}} e^{-\Omega t}
\end{align*}
The constant in front of the exponential is so that the income in the first year following a unit shock will be equal to one.

The flow of income at a point in time $s$ is therefore:
\begin{align*}
y(t) = \frac{\Omega}{1-e^{-\Omega}} \int_{-\infty}^{t} e^{-\Omega (t-s)}dQ_s
\end{align*}
Observed income over the year $T$ is the integral of the income flow over that year:
\begin{align*}
y_T^{obs} &=\frac{\Omega}{1-e^{-\Omega}} \int_{T-1}^{T} \int_{-\infty}^{t} e^{-\Omega (t-s)}dQ_s dt \\
&= \frac{\Omega}{1-e^{-\Omega}} \left[ \int_{T-1}^{T} \int_{-\infty}^{T-1} e^{-\Omega (t-s)}dQ_s dt + \int_{T-1}^{T} \int_{T-1}^{t} e^{-\Omega (t-s)}dQ_s dt \right] 
\end{align*}
Swapping the order of the integrals gives:
\begin{align*}
y_T^{obs} &=\frac{\Omega}{1-e^{-\Omega}} \left[ \int_{-\infty}^{T-1} \int_{T-1}^{T} e^{-\Omega (t-s)} dt dQ_s  + \int_{T-1}^{T} \int_{s}^{T} e^{-\Omega (t-s)} dt dQ_s \right] \\
&=\frac{1}{1-e^{-\Omega}} \left[ \int_{-\infty}^{T-1} ( e^{-\Omega (T-1-s)} - e^{-\Omega (T-s)}) dQ_s  + \int_{T-1}^{T}  ( 1 - e^{-\Omega (T-s)} ) dQ_s \right] \\
&= \frac{1}{1-e^{-\Omega}}\int_{T-1}^{T}  ( 1 - e^{-\Omega (T-s)} ) dQ_s  + \int_{-\infty}^{T-1}  e^{-\Omega (T-1-s)} dQ_s  
\end{align*}
Now take the first difference:
\begin{align*}
\Delta y_T^{obs} &=\frac{1}{1-e^{-\Omega}}\int_{T-1}^{T}  ( 1 - e^{-\Omega (T-s)} ) dQ_s \\ & \qquad + \int_{T-2}^{T-1} \left( e^{-\Omega (T-1-s)}   - \frac{1}{1-e^{-\Omega}}  ( 1 - e^{-\Omega (T-1-s)} ) \right) dQ_s \\
& \qquad - \int_{-\infty}^{T-2}  e^{-\Omega (T-2-s)} (1-e^{-\Omega}) dQ_s  \\
&=\frac{1}{1-e^{-\Omega}}\int_{T-1}^{T}  ( 1 - e^{-\Omega (T-s)} ) dQ_s \\ & \qquad +  \frac{1}{1-e^{-\Omega}} \int_{T-2}^{T-1}\left( (2-e^{-\Omega}) e^{-\Omega (T-1-s)} -1  \right)dQ_s \\
& \qquad -  (1-e^{-\Omega}) \int_{-\infty}^{T-2}  e^{-\Omega (T-2-s)} dQ_s 
\end{align*}
Calculate covariances - first the variance:
%%%%%%%%%%%%%%%%%%%%%%%%%%%%%%%%%%%%%%%%%%%%%%%%%%%%%%%%%%%%
% Var(Delta Y_T)
\begin{align*}
\mathrm{Var}(\Delta y_T^{obs}) &= \frac{1}{(1-e^{-\Omega})^2}\int_{T-1}^{T} ( 1 - 2e^{-\Omega (T-s)}  + e^{-2\Omega (T-s)}) ds \\
& \qquad + \frac{1}{(1-e^{-\Omega})^2} \int_{T-2}^{T-1}\left( (2-e^{-\Omega})^2 e^{-2\Omega (T-1-s)} -2(2-e^{-\Omega}) e^{-\Omega (T-1-s)} +1  \right)ds \\
& \qquad + (1-e^{-\Omega})^2 \int_{-\infty}^{T-2}  e^{-2\Omega (T-2-s)} ds \\
&= \frac{1}{(1-e^{-\Omega})^2}\left( 1 - \frac{2}{\Omega}(1-e^{-\Omega})    + \frac{1}{2\Omega}(1-e^{-2\Omega}) \right) \\
& \qquad + \frac{1}{(1-e^{-\Omega})^2}\left( (2-e^{-\Omega})^2\frac{1}{2\Omega}(1-e^{-2\Omega}) -2(2-e^{-\Omega}) \frac{1}{\Omega}(1-e^{-\Omega}) +1  \right) \\
& \qquad +  \frac{1}{2\Omega}(1-e^{-\Omega})^2 \\
&= \frac{1}{(1-e^{-\Omega})^2}\left( 2    + \left( (2-e^{-\Omega})^2 +1\right)\frac{1}{2\Omega}(1-e^{-2\Omega}) -(3-e^{-\Omega}) \frac{2}{\Omega}(1-e^{-\Omega})   \right) \\
& \qquad +  \frac{1}{2\Omega}(1-e^{-\Omega})^2 \\
&=  \frac{1}{(1-e^{-\Omega})^2}\left( 2   - \frac{1}{2\Omega}\left( 7 -12e^{-\Omega} +8e^{-2\Omega} -4e^{-3\Omega} + e^{-4\Omega} \right) \right) \\
& \qquad +  \frac{1}{2\Omega}(1-e^{-\Omega})^2 \\
&=  \frac{1}{(1-e^{-\Omega})^2}\left( 2   - \frac{1}{\Omega}\left( 3 -4e^{-\Omega} +e^{-2\Omega}  \right) \right)\\
&=  \frac{2}{(1-e^{-\Omega})^2}   - \frac{ 3 -e^{-\Omega}}{\Omega(1-e^{-\Omega})}
\end{align*}


%%%%%%%%%%%%%%%%%%%%%%%%%%%%%%%%%%%%%%%%%%%%%%%%%%%%%%%%%%%%
% Cov(Delta Y_T,Delta Y_T-1)
Next calculate covariance with one lag:
\begin{align*}
\mathrm{Cov}(\Delta y_T^{obs},\Delta y_{T-1}^{obs}) 
&=\frac{1}{(1-e^{-\Omega})^2}\int_{T-2}^{T-1}  ( 1 - e^{-\Omega (T-1-s)} ) \left( (2-e^{-\Omega}) e^{-\Omega (T-1-s)} -1  \right) ds \\
& \qquad - \int_{T-3}^{T-2}\left( (2-e^{-\Omega}) e^{-\Omega (T-2-s)} -1  \right) e^{-\Omega (T-2-s)} ds \\
& \qquad +  (1-e^{-\Omega})^2 \int_{-\infty}^{T-3}  e^{-\Omega (T-3-s)}  e^{-\Omega (T-2-s)} dQ_s \\
&=   \frac{1}{2\Omega}(2-e^{-\Omega}) - \frac{1}{(1-e^{-\Omega})^2}( 1 - \frac{1-e^{-\Omega}}{\Omega}) \\
& \qquad - \frac{1-e^{-2\Omega}}{2\Omega}\left( 2- e^{-\Omega} \right) + \frac{1}{\Omega}(1-e^{-\Omega}) \\
& \qquad +  \frac{1 }{2\Omega} e^{-\Omega}(1-e^{-\Omega})^2 \\
&=   \frac{1}{2\Omega}(2-e^{-\Omega}) - \frac{1}{(1-e^{-\Omega})^2}( 1 - \frac{1-e^{-\Omega}}{\Omega})
\end{align*}





%%%%%%%%%%%%%%%%%%%%%%%%%%%%%%%%%%%%%%%%%%%%%%%%%%%%%%%%%%%%
% Cov(Delta Y_T,Delta Y_T-M)
And the covariance with $M \geq 2$ lags: 
\begin{align*}
\mathrm{Cov}(\Delta y_T^{obs},\Delta y_{T-M}^{obs}) 
&=-\int_{T-M-1}^{T-M}  ( 1 - e^{-\Omega (T-M-s)} ) e^{-\Omega (T-2-s)} ds \\ 
& \qquad - \int_{T-M-2}^{T-M-1}\left( (2-e^{-\Omega}) e^{-\Omega (T-M-1-s)} -1  \right)e^{-\Omega (T-2-s)} ds \\
& \qquad +  (1-e^{-\Omega})^2 \int_{-\infty}^{T-M-2}  e^{-\Omega (T-M-2-s)} e^{-\Omega (T-2-s)} ds \\
&= -\frac{1}{\Omega}(1-e^{-\Omega})e^{-\Omega (M-2)} +\frac{1}{2\Omega}(1-e^{-2\Omega})e^{-\Omega (M-2)} \\
& \qquad - (2-e^{-\Omega}) e^{-\Omega(M-1)}\frac{1}{2\Omega}(1-e^{-2\Omega}) + \frac{1}{\Omega} e^{-\Omega(M-1)} (1-e^{-\Omega})\\
& \qquad +(1-e^{-\Omega})^2 \frac{1}{2\Omega} e^{-\Omega M} \\
\end{align*}


%%%%%%%%%%%%%%%%%%%%%%%%%%%%%%%%%%%%%%%%%%%%%%%%%%%%%%%%%%%%
%%%%%%%%%%%%%%%%%%%%%%%%%%%%%%%%%%%%%%%%%%%%%%%%%%%%%%%%%%%%
% Var( Y_T)
Note the variance of $y_T^{obs}$ is not equal to one (as in the discrete time case). For comparison I calculate it here:
\begin{align*}
\mathrm{Var}( y_T^{obs}) &= \frac{1}{(1-e^{-\Omega})^2}\int_{T-1}^{T} ( 1 - 2e^{-\Omega (T-s)}  + e^{-2\Omega (T-s)}) ds \\
& \qquad +  \int_{-\infty}^{T-1}  e^{-2\Omega (T-1-s)} ds \\
&= \frac{1}{(1-e^{-\Omega})^2}\left( 1 - \frac{2}{\Omega}(1-e^{-\Omega})    + \frac{1}{2\Omega}(1-e^{-2\Omega}) \right) +  \frac{1}{2\Omega}
\end{align*}
\textit{Exponentially Decaying Consumption Process}\\
Consumption responds to a transitory income shock according to the function:
\begin{align*}
g(t) = \frac{\psi \theta}{1-e^{-\theta}} e^{-\theta t}
\end{align*}
The flow of consumption is observed at the end of each calendar year:
\begin{align*}
c_T^{obs} = \frac{\psi \theta}{1-e^{-\theta}} \int_{-\infty}^{T} e^{-\theta (T-s)}dQ_s
\end{align*}
Now take the first difference
\begin{align*}
\Delta c_T^{obs} &= \frac{ \psi\theta}{1-e^{-\theta}} \left[ \int_{T-1}^{T} e^{-\theta (T-s)}dQ_s +  \int_{-\infty}^{T-1} e^{-\theta (T-s)} - e^{-\theta (T-1-s)}dQ_s \right] \\
&= \frac{ \psi \theta}{1-e^{-\theta}}  \int_{T-1}^{T} e^{-\theta (T-s)}dQ_s - \psi \theta \int_{-\infty}^{T-1} e^{-\theta (T-1-s)} dQ_s
\end{align*}
Calculate covariances:
%%%%%%%%%%%%%%%%%%%%%%%%%%%%%%%%%%%%%%%%%%%%%%%%%%%%%%%%%%%%
% Var(Delta C_T)
\begin{align*}
\mathrm{Var}(\Delta c_T^{obs}) &= \frac{ \psi^2\theta^2}{(1-e^{-\theta})^2}  \int_{T-1}^{T} e^{-2\theta (T-s)}ds + \psi^2\theta^2 \int_{-\infty}^{T-1} e^{-2\theta (T-1-s)} ds \\
&= \frac{ \psi^2\theta}{2} \left(1+ \frac{ 1-e^{-2\theta}}{(1-e^{-\theta})^2} \right)\\
&= \frac{ \psi^2\theta}{1-e^{-\theta}}
\end{align*}
%%%%%%%%%%%%%%%%%%%%%%%%%%%%%%%%%%%%%%%%%%%%%%%%%%%%%%%%%%%%
% Cov(Delta C_T,Delta C_T-M)
\begin{align*}
\mathrm{Cov}(\Delta c_T^{obs},\Delta c_{T-M}^{obs}) &= \frac{ -\psi^2\theta^2}{(1-e^{-\theta})}  \int_{T-M-1}^{T-M} e^{-\theta M}e^{-\theta (2(T-M-s)-1) }ds \\
& \qquad + \psi^2\theta^2 \int_{-\infty}^{T-M-1} e^{-\theta M} e^{-2\theta (T-M-1-s)} ds \\
&= \frac{\psi^2\theta}{2} e^{-\theta (M-1)} \left[ \frac{ e^{-2\theta}-1 }{1-e^{-\theta} }   + e^{-\theta}  \right] \\
&= \frac{-\psi^2\theta}{2} e^{-\theta (M-1)} 
\end{align*}

\textit{Covariance of Income and Consumption}\\
%%%%%%%%%%%%%%%%%%%%%%%%%%%%%%%%%%%%%%%%%%%%%%%%%%%%%%%%%%%%
% Cov(Delta C_T,Delta Y_T)
\begin{align*}
\mathrm{Cov}(\Delta c_T^{obs},\Delta y_{T}^{obs}) 
&=\frac{\psi \theta}{(1-e^{-\Omega})(1-e^{-\theta})}\int_{T-1}^{T}  ( 1 - e^{-\Omega (T-s)} ) e^{-\theta (T-s)}ds \\
& \qquad  - \frac{\psi \theta}{1-e^{-\Omega}} \int_{T-2}^{T-1}\left( (2-e^{-\Omega}) e^{-\Omega (T-1-s)} -1  \right)  e^{-\theta (T-1-s)}ds \\
& \qquad + \psi \theta (1-e^{-\Omega}) \int_{-\infty}^{T-2}  e^{-\Omega (T-2-s)}  e^{-\theta (T-1-s)}ds \\
&=\frac{\psi \theta}{(1-e^{-\Omega})(1-e^{-\theta})} \left[ \frac{1}{\theta}(1-e^{-\theta}) - \frac{1}{\Omega + \theta}(1-e^{-(\Omega+\theta)})  \right] \\
& \qquad  - \frac{\psi \theta}{1-e^{-\Omega}} \left[ (2-e^{-\Omega}) \frac{1}{\Omega + \theta}(1-e^{-(\Omega+\theta)}) - \frac{1}{\theta}(1-e^{-\theta})  \right] \\
& \qquad + \psi \theta (1-e^{-\Omega}) e^{-\theta} \frac{1}{\Omega + \theta} \\
\end{align*}

%%%%%%%%%%%%%%%%%%%%%%%%%%%%%%%%%%%%%%%%%%%%%%%%%%%%%%%%%%%%
% Cov(Delta C_T,Delta Y_{T+1})
\begin{align*}
\mathrm{Cov}(\Delta c_{T}^{obs},\Delta y_{T+1}^{obs}) 
&= \frac{\psi \theta}{(1-e^{-\Omega})(1-e^{-\theta})} \int_{T-1}^{T}\left( (2-e^{-\Omega}) e^{-\Omega (T-s)} -1  \right) e^{-\theta (T-s)}ds \\
& \qquad +\psi \theta  (1-e^{-\Omega}) \int_{-\infty}^{T-1}  e^{-\Omega (T-1-s)} e^{-\theta (T-1-s)}ds \\
&= \frac{\psi \theta}{(1-e^{-\Omega})(1-e^{-\theta})} \left[ (2-e^{-\Omega})\frac{1}{\Omega + \theta}(1-e^{-(\Omega+\theta)}) - \frac{1}{\theta}(1-e^{-\theta})  \right] \\
& \qquad +\psi \theta  (1-e^{-\Omega}) \frac{1}{\Omega + \theta}
\end{align*}

%%%%%%%%%%%%%%%%%%%%%%%%%%%%%%%%%%%%%%%%%%%%%%%%%%%%%%%%%%%%
% Cov(Delta C_T,Delta Y_{T+M}) 
For $M \geq 2$:
\begin{align*}
\mathrm{Cov}(\Delta c_T^{obs},\Delta y_{T+M}^{obs}) 
&= -\psi \theta\frac{1-e^{-\Omega}}{1-e^{-\theta}} e^{-\Omega (M-2)}\int_{T-1}^{T}  e^{-\Omega (T-s)} e^{-\theta (T-s)}ds \\
& \qquad + \psi \theta (1-e^{-\Omega})e^{-\Omega(M-1)} \int_{-\infty}^{T-1}  e^{-\Omega (T-1-s)} e^{-\theta (T-1-s)}ds \\
&= -\psi \theta\frac{1-e^{-\Omega}}{1-e^{-\theta}} e^{-\Omega (M-2)}\frac{1}{\Omega + \theta}(1-e^{-(\Omega+\theta)}) \\
& \qquad + \psi \theta (1-e^{-\Omega})e^{-\Omega(M-1)} \frac{1}{\Omega + \theta}
\end{align*}

For $M \geq 1$
%%%%%%%%%%%%%%%%%%%%%%%%%%%%%%%%%%%%%%%%%%%%%%%%%%%%%%%%%%%%
% Cov(Delta C_T,Delta Y_{T-M})
\begin{align*}
\mathrm{Cov}(\Delta c_{T+M}^{obs},\Delta y_{T}^{obs}) 
&=-\frac{\psi \theta}{1-e^{-\Omega}} e^{-\theta (M-1)} \int_{T-1}^{T}  ( 1 - e^{-\Omega (T-s)} ) e^{-\theta (T-s)}ds \\
& \qquad +  -\frac{\psi \theta}{1-e^{-\Omega}}  e^{-\theta M} \int_{T-2}^{T-1}\left( (2-e^{-\Omega}) e^{-\Omega (T-1-s)} -1  \right)e^{-\theta (T-1-s)}ds \\
& \qquad +  \psi \theta (1-e^{-\Omega}) e^{-\theta (M+1)} \int_{-\infty}^{T-2}  e^{-\Omega (T-2-s)} e^{-\theta (T-2-s)}ds \\
&=-\frac{\psi \theta}{1-e^{-\Omega}} e^{-\theta (M-1)} \left[ \frac{1}{\theta}(1-e^{-\theta}) - \frac{1}{\Omega + \theta}(1-e^{-(\Omega+\theta)})  \right]\\
& \qquad +  -\frac{\psi \theta}{1-e^{-\Omega}}  e^{-\theta M} \left[ (2-e^{-\Omega})\frac{1}{\Omega + \theta}(1-e^{-(\Omega+\theta)})  - \frac{1}{\theta}(1-e^{-\theta})\right] \\
& \qquad +  \psi \theta (1-e^{-\Omega}) e^{-\theta (M+1)}\frac{1}{\Omega + \theta}
\end{align*}



