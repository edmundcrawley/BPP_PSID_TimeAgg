% !TeX spellcheck = en_GB
\documentclass[12pt,pdftex,letterpaper]{article}
%            \usepackage{setspace}
\usepackage[dvips,]{graphicx} %draft option suppresses graphics dvi display
%            \usepackage{lscape}
%            \usepackage{latexsym}
%            \usepackage{endnotes}
%            \usepackage{epsfig}
%           \singlespace
\setlength{\textwidth}{6.5in}
\setlength{\textheight}{9in}
\addtolength{\topmargin}{-\topmargin} 
\setlength{\oddsidemargin}{0in}
\setlength{\evensidemargin}{0in}
\addtolength{\headsep}{-\headsep}
\addtolength{\topskip}{-\topskip}
\addtolength{\headheight}{-\headheight}
\setcounter{secnumdepth}{2}
%            \renewcommand{\thesection}{\arabic{section}}
% \renewcommand{\footnote}{\endnote}
\newtheorem{proposition}{Proposition}
\newtheorem{definition}{Definition}
\newtheorem{lemma}{lemma}
\newtheorem{corollary}{Corollary}
\newtheorem{assumption}{Assumption}
\newcommand{\Prob}{\operatorname{Prob}}
\clubpenalty 5000
\widowpenalty 5000
\renewcommand{\baselinestretch}{1.25}
\usepackage{amsmath}
\usepackage{amsthm}
\usepackage{amsfonts}
\usepackage{amssymb}
\usepackage{bbm}
\newcommand{\N}{\mathbb{N}}
\newcommand{\R}{\mathbb{R}}
\newcommand{\E}{\mathbb{E}}

\begin{document}


\subsubsection{Moments in the Exponential Decay Model} \label{exp_decay_moments_appendix}

\textit{Exponentially Decaying Income Process}\\

A shock to transitory income decays exponentially according to the function:
\begin{align*}
f(t) = \frac{\Omega}{1-e^{-\Omega}} e^{-\Omega t}
\end{align*}
The constant in front of the exponential is so that the income in the first year following a unit shock will be equal to one.

The flow of income at a point in time $s$ is therefore:
\begin{align*}
y(t) = \frac{\Omega}{1-e^{-\Omega}} \int_{-\infty}^{t} e^{-\Omega (t-s)}dQ_s
\end{align*}
Observed income over the year $T$ is the integral of the income flow over that year:
\begin{align*}
y_T^{obs} &=\frac{\Omega}{1-e^{-\Omega}} \int_{T-1}^{T} \int_{-\infty}^{t} e^{-\Omega (t-s)}dQ_s dt \\
&= \frac{\Omega}{1-e^{-\Omega}} \left[ \int_{T-1}^{T} \int_{-\infty}^{T-1} e^{-\Omega (t-s)}dQ_s dt + \int_{T-1}^{T} \int_{T-1}^{t} e^{-\Omega (t-s)}dQ_s dt \right] 
\end{align*}
Swapping the order of the integrals gives:
\begin{align*}
y_T^{obs} &=\frac{\Omega}{1-e^{-\Omega}} \left[ \int_{-\infty}^{T-1} \int_{T-1}^{T} e^{-\Omega (t-s)} dt dQ_s  + \int_{T-1}^{T} \int_{s}^{T} e^{-\Omega (t-s)} dt dQ_s \right] \\
&=\frac{1}{1-e^{-\Omega}} \left[ \int_{-\infty}^{T-1} ( e^{-\Omega (T-1-s)} - e^{-\Omega (T-s)}) dQ_s  + \int_{T-1}^{T}  ( 1 - e^{-\Omega (T-s)} ) dQ_s \right] \\
&= \frac{1}{1-e^{-\Omega}}\int_{T-1}^{T}  ( 1 - e^{-\Omega (T-s)} ) dQ_s  + \int_{-\infty}^{T-1}  e^{-\Omega (T-1-s)} dQ_s  
\end{align*}
Now take the first difference:
\begin{align*}
\Delta y_T^{obs} &=\frac{1}{1-e^{-\Omega}}\int_{T-1}^{T}  ( 1 - e^{-\Omega (T-s)} ) dQ_s \\ & \qquad + \int_{T-2}^{T-1} \left( e^{-\Omega (T-1-s)}   - \frac{1}{1-e^{-\Omega}}  ( 1 - e^{-\Omega (T-1-s)} ) \right) dQ_s \\
& \qquad - \int_{-\infty}^{T-2}  e^{-\Omega (T-2-s)} (1-e^{-\Omega}) dQ_s  \\
&=\frac{1}{1-e^{-\Omega}}\int_{T-1}^{T}  ( 1 - e^{-\Omega (T-s)} ) dQ_s \\ & \qquad +  \frac{1}{1-e^{-\Omega}} \int_{T-2}^{T-1}\left( (2-e^{-\Omega}) e^{-\Omega (T-1-s)} -1  \right)dQ_s \\
& \qquad -  (1-e^{-\Omega}) \int_{-\infty}^{T-2}  e^{-\Omega (T-2-s)} dQ_s 
\end{align*}
Calculate covariances - first the variance:
%%%%%%%%%%%%%%%%%%%%%%%%%%%%%%%%%%%%%%%%%%%%%%%%%%%%%%%%%%%%
% Var(Delta Y_T)
\begin{align*}
\mathrm{Var}(\Delta y_T^{obs}) &= \frac{1}{(1-e^{-\Omega})^2}\int_{T-1}^{T} ( 1 - 2e^{-\Omega (T-s)}  + e^{-2\Omega (T-s)}) ds \\
& \qquad + \frac{1}{(1-e^{-\Omega})^2} \int_{T-2}^{T-1}\left( (2-e^{-\Omega})^2 e^{-2\Omega (T-1-s)} -2(2-e^{-\Omega}) e^{-\Omega (T-1-s)} +1  \right)ds \\
& \qquad + (1-e^{-\Omega})^2 \int_{-\infty}^{T-2}  e^{-2\Omega (T-2-s)} ds \\
&= \frac{1}{(1-e^{-\Omega})^2}\left( 1 - \frac{2}{\Omega}(1-e^{-\Omega})    + \frac{1}{2\Omega}(1-e^{-2\Omega}) \right) \\
& \qquad + \frac{1}{(1-e^{-\Omega})^2}\left( (2-e^{-\Omega})^2\frac{1}{2\Omega}(1-e^{-2\Omega}) -2(2-e^{-\Omega}) \frac{1}{\Omega}(1-e^{-\Omega}) +1  \right) \\
& \qquad +  \frac{1}{2\Omega}(1-e^{-\Omega})^2 \\
&= \frac{1}{(1-e^{-\Omega})^2}\left( 2    + \left( (2-e^{-\Omega})^2 +1\right)\frac{1}{2\Omega}(1-e^{-2\Omega}) -(3-e^{-\Omega}) \frac{2}{\Omega}(1-e^{-\Omega})   \right) \\
& \qquad +  \frac{1}{2\Omega}(1-e^{-\Omega})^2 \\
&=  \frac{1}{(1-e^{-\Omega})^2}\left( 2   - \frac{1}{2\Omega}\left( 7 -12e^{-\Omega} +8e^{-2\Omega} -4e^{-3\Omega} + e^{-4\Omega} \right) \right) \\
& \qquad +  \frac{1}{2\Omega}(1-e^{-\Omega})^2 \\
&=  \frac{1}{(1-e^{-\Omega})^2}\left( 2   - \frac{1}{\Omega}\left( 3 -4e^{-\Omega} +e^{-2\Omega}  \right) \right)\\
&=  \frac{2}{(1-e^{-\Omega})^2}   - \frac{ 3 -e^{-\Omega}}{\Omega(1-e^{-\Omega})}
\end{align*}


%%%%%%%%%%%%%%%%%%%%%%%%%%%%%%%%%%%%%%%%%%%%%%%%%%%%%%%%%%%%
% Cov(Delta Y_T,Delta Y_T-1)
Next calculate covariance with one lag:
\begin{align*}
\mathrm{Cov}(\Delta y_T^{obs},\Delta y_{T-1}^{obs}) 
&=\frac{1}{(1-e^{-\Omega})^2}\int_{T-2}^{T-1}  ( 1 - e^{-\Omega (T-1-s)} ) \left( (2-e^{-\Omega}) e^{-\Omega (T-1-s)} -1  \right) ds \\
& \qquad - \int_{T-3}^{T-2}\left( (2-e^{-\Omega}) e^{-\Omega (T-2-s)} -1  \right) e^{-\Omega (T-2-s)} ds \\
& \qquad +  (1-e^{-\Omega})^2 \int_{-\infty}^{T-3}  e^{-\Omega (T-3-s)}  e^{-\Omega (T-2-s)} dQ_s \\
&=   \frac{1}{2\Omega}(2-e^{-\Omega}) - \frac{1}{(1-e^{-\Omega})^2}( 1 - \frac{1-e^{-\Omega}}{\Omega}) \\
& \qquad - \frac{1-e^{-2\Omega}}{2\Omega}\left( 2- e^{-\Omega} \right) + \frac{1}{\Omega}(1-e^{-\Omega}) \\
& \qquad +  \frac{1 }{2\Omega} e^{-\Omega}(1-e^{-\Omega})^2 \\
&=   \frac{1}{2\Omega}(2-e^{-\Omega}) - \frac{1}{(1-e^{-\Omega})^2}( 1 - \frac{1-e^{-\Omega}}{\Omega})
\end{align*}





%%%%%%%%%%%%%%%%%%%%%%%%%%%%%%%%%%%%%%%%%%%%%%%%%%%%%%%%%%%%
% Cov(Delta Y_T,Delta Y_T-M)
And the covariance with $M \geq 2$ lags: 
\begin{align*}
\mathrm{Cov}(\Delta y_T^{obs},\Delta y_{T-M}^{obs}) 
&=-\int_{T-M-1}^{T-M}  ( 1 - e^{-\Omega (T-M-s)} ) e^{-\Omega (T-2-s)} ds \\ 
& \qquad - \int_{T-M-2}^{T-M-1}\left( (2-e^{-\Omega}) e^{-\Omega (T-M-1-s)} -1  \right)e^{-\Omega (T-2-s)} ds \\
& \qquad +  (1-e^{-\Omega})^2 \int_{-\infty}^{T-M-2}  e^{-\Omega (T-M-2-s)} e^{-\Omega (T-2-s)} ds \\
&= -\frac{1}{\Omega}(1-e^{-\Omega})e^{-\Omega (M-2)} +\frac{1}{2\Omega}(1-e^{-2\Omega})e^{-\Omega (M-2)} \\
& \qquad - (2-e^{-\Omega}) e^{-\Omega(M-1)}\frac{1}{2\Omega}(1-e^{-2\Omega}) + \frac{1}{\Omega} e^{-\Omega(M-1)} (1-e^{-\Omega})\\
& \qquad +(1-e^{-\Omega})^2 \frac{1}{2\Omega} e^{-\Omega M} \\
\end{align*}


%%%%%%%%%%%%%%%%%%%%%%%%%%%%%%%%%%%%%%%%%%%%%%%%%%%%%%%%%%%%
%%%%%%%%%%%%%%%%%%%%%%%%%%%%%%%%%%%%%%%%%%%%%%%%%%%%%%%%%%%%
% Var( Y_T)
Note the variance of $y_T^{obs}$ is not equal to one (as in the discrete time case). For comparison I calculate it here:
\begin{align*}
\mathrm{Var}( y_T^{obs}) &= \frac{1}{(1-e^{-\Omega})^2}\int_{T-1}^{T} ( 1 - 2e^{-\Omega (T-s)}  + e^{-2\Omega (T-s)}) ds \\
& \qquad +  \int_{-\infty}^{T-1}  e^{-2\Omega (T-1-s)} ds \\
&= \frac{1}{(1-e^{-\Omega})^2}\left( 1 - \frac{2}{\Omega}(1-e^{-\Omega})    + \frac{1}{2\Omega}(1-e^{-2\Omega}) \right) +  \frac{1}{2\Omega}
\end{align*}
\textit{Exponentially Decaying Consumption Process}\\
Consumption responds to a transitory income shock according to the function:
\begin{align*}
g(t) = \frac{\psi \theta}{1-e^{-\theta}} e^{-\theta t}
\end{align*}
Consumption is time aggregated. In exactly parallel calculations as for income, the observed change in consumption is:
\begin{align*}
\Delta c_T^{obs} 
&=\frac{\psi}{1-e^{-\theta}}\int_{T-1}^{T}  ( 1 - e^{-\theta (T-s)} ) dQ_s \\ 
& \qquad +  \frac{\psi}{1-e^{-\theta}} \int_{T-2}^{T-1}\left( (2-e^{-\theta}) e^{-\theta (T-1-s)} -1  \right)dQ_s \\
& \qquad - \psi (1-e^{-\theta}) \int_{-\infty}^{T-2}  e^{-\theta (T-2-s)} dQ_s 
\end{align*}
The covariance matrix with itself is also parallel with income:
%%%%%%%%%%%%%%%%%%%%%%%%%%%%%%%%%%%%%%%%%%%%%%%%%%%%%%%%%%%%
% Var(Delta C_T)
\begin{align*}
\mathrm{Var}(\Delta c_T^{obs}) 
&=  \frac{2}{(1-e^{-\theta})^2}   - \frac{ 3 -e^{-\theta}}{\theta(1-e^{-\theta})}
\end{align*}


%%%%%%%%%%%%%%%%%%%%%%%%%%%%%%%%%%%%%%%%%%%%%%%%%%%%%%%%%%%%
% Cov(Delta C_T,Delta C_T-1)
Next calculate covariance with one lag:
\begin{align*}
\mathrm{Cov}(\Delta c_T^{obs},\Delta c_{T-1}^{obs}) 
&=   \frac{1}{2\theta}(2-e^{-\theta}) - \frac{1}{(1-e^{-\theta})^2}( 1 - \frac{1-e^{-\theta}}{\theta})
\end{align*}

%%%%%%%%%%%%%%%%%%%%%%%%%%%%%%%%%%%%%%%%%%%%%%%%%%%%%%%%%%%%
% Cov(Delta C_T,Delta C_T-M)
And the covariance with $M \geq 2$ lags: 
\begin{align*}
\mathrm{Cov}(\Delta c_T^{obs},\Delta c_{T-M}^{obs}) 
&= -\frac{1}{\theta}(1-e^{-\theta})e^{-\theta (M-2)} +\frac{1}{2\theta}(1-e^{-2\theta})e^{-\theta (M-2)} \\
& \qquad - (2-e^{-\theta}) e^{-\theta(M-1)}\frac{1}{2\theta}(1-e^{-2\theta}) + \frac{1}{\theta} e^{-\theta(M-1)} (1-e^{-\theta})\\
& \qquad +(1-e^{-\theta})^2 \frac{1}{2\theta} e^{-\theta M} \\
\end{align*}


%%%%%%%%%%%%%%%%%%%%%%%%%%%%%%%%%%%%%%%%%%%%%%%%%%%%%%%%%%%%
%%%%%%%%%%%%%%%%%%%%%%%%%%%%%%%%%%%%%%%%%%%%%%%%%%%%%%%%%%%%


\end{document}

